   \xname{GridsAlgorithmsAndReuse}
  \begin{Label}{GridsAlgorithmsAndReuse}
    \introsection{Grids, Algorithms and Reuse}
  \end{Label}

  Grids are geometric data structures for representing certain subdivisions of space.
  They are used in numerical solution of partial differential equations
  (finite element or finite volume methods),
  computer graphics, computational geometry, geometric modeling, 
  or geographic information systems.
  In each of these areas, they are known by differing names,
  like mesh, subdivision, triangulation, cell or simplicial complex, 
  BRep, or planar straight line graph (PSLG), to name just a few.

  There is, however, a common underlying mathematical concept unifying (most of)
  these different notions, namely the 
  \xlink{CW-complex}{http://www.treasure-troves.com/math/CW-Complex.html}
  of combinatorial topology.

  Software development in the domain of geometric computing is 
  notoriously hard, 
  and reuse of algorithms is particularly difficult to achieve.
  This stems from the literally thousand of possibilities of
  representing such grids which are found throughout geometric software.
  Therefore, implementations of algorithms are tied very closely to the
  concrete data structure they are written for.
  As an example, consider the following algorithm,
  which calculates the surface of each cell in a grid 
  (e.g. for 2D triangulation, this is the perimeter of each triangle in the grid):

   \begin{figure}[h]
     \begin{center}
       \begin{Label}{algo-surface}
       \T\includegraphics{bilder/algo-surface}
       \W\htmlimage[ALT="A simple algorithm calculating cell surfaces"]{%
          \img{algo-surface}}
        \end{Label}
     \end{center}
   \end{figure}

  Here ${\cal G}$ is a grid of dimension $d$, ${\cal G}^d$ is the set of its cells
  (elements of dimension $d$), and a \emph{facet} is an element of dimension $d-1$.

  This algorithm is completely independent of any concrete grid representation;
  it works for any dimension, for triangles as well as for general polygons as cells,
  and it does not care whether cells are stored in arrays or lists, or even given
  implicitely as in a cartesian grid.

  Now consider what happens when we implement this algorithm for a given grid data structure.
  Let us assume we have a simple data structure for a 2D triangulation
  where cell-vertex incidences are stored in a plain array \texttt{cells},
  such that \texttt{cells[3*c +v]} gives the index of the \texttt{v}'th vertex
  of cell \texttt{c}. Likewise, the array \texttt{geom} is assumed to 
  hold the coordinates of vertex \texttt{v}.
  A possible implementation could be the following:
  \begin{example}  
   double * surf = new double[nc];
   for(c = 0; c < nc; ++c) {
     surf[c] = 0.0;
     for(vc = 0; vc < 3; ++vc) {
       int vc1 = (vc+1)\%3;
       double dx = (geom[2*cells[3*c+vc ]   ] - geom[2*cells[3*c+vc1]   ]);
       double dy = (geom[2*cells[3*c+vc ]+1 ] - geom[2*cells[3*c+vc1] +1]);
       surf[c] +=  sqrt(dx*dx+dy*dy);
     }
   }
  \end{example}
  
  Evidently, any generality is lost. 
  This is perhaps not an issue for this trivial algorithm, 
  but most algorithm are somewhat more complex.
  And there is a lot of algorithms operating on grids 
  one might want to use.
  So \ldots what can one do? 
  The idea is to exploit the common underlying structure of grids
  in order to create an abstract grid interface.
  Algorithms will be implemented on top of this interface,
  much like STL algorithms are implemented on top of iterators
  and do not use directly the way data is stored e.g. in a list.

