\xname{GridElement}    
\begin{Label}{GridElement}
\conceptsection{Grid Element Concept}
\end{Label}

\conceptsubsection{Description}

A {\em Grid Element\/} is an entity, such as a 
\sectionlink{\concept{Grid Vertex}}{GridVertex},
that belongs to a {\em Grid\/}.
To each grid element, there is associated a unique grid (the {\em anchor\/} grid). 
Two elements may be compared for
equality, if they belong to the same grid.
    
Conceptually, a combinatorial grid consists of its elements of different dimension
(Vertices, Edges and so on), plus an incidence relation between them. 
This does not imply, however, that the element constituing a grid must be stored
permanently within the grid.

We name the element types of a grid consistently
according to the following table, where we
distinguish between names relating to element dimension and element
codimension\noteref{note-dim-codim}
    
\begin{tabular}{lcc}
  \\ 
  \hline
  \bf  k-Element &  \bf  Dimension &  \bf  Codimension  \\ 
  \hline
  {\tt Vertex} &   0 &   ~  \\ 
  {\tt Edge} &   1 &   ~  \\ 
  {\tt Face}\noteref{note-face}  &   2 &   ~  \\ 
  {\tt Facet} &   ~ &   1  \\ 
  {\tt Cell} &   ~ &   0   \\ 
  \hline
  \\
\end{tabular}
    
This naming scheme allows for a dimension-independent
formulation of many algorithms: 
for \link{example}{flux-example} fluxes in finite volume algorithms
are always defined on Facets.
      
\conceptsubsection{Refinement of}
\sectionlink{\concept{Grid Entity}}{GridEntity}
    
The only refinement is, in fact, that the anchor type is equal to the grid type.

\conceptsubsection{Notation}
{\tt E} is a type which is a model of {\em Grid Element\/}
\\
{\tt e, e1, e2} are objects of type {\tt E}
\\
{\tt g} is an object of type {\tt E::grid\_type}

\conceptsubsection{Associated types}

\begin{tabularx}{12cm}{llX}
  \\ 
  \hline
  \bf  Name  &\bf  Expression  &\bf  Description   \\ 
  \hline
  Grid type &  {\tt E::grid\_type} &
  the same as {\tt E::anchor\_type}
  (defined in \sectionlink{\concept{Grid Entity}}{GridEntity})  
  \\ 
  \hline
  \\
\end{tabularx}

\conceptsubsection{Valid Expressions}
\begin{tabular}{llll}
  \\  
  \hline
  \bf  Name  &\bf  Expression  &\bf  Type requirements  & \bf  return type  \\ 
  \hline
  Anchor  Grid &  {\tt e.TheGrid()} &  ~ &  {\tt grid\_type const\&}  \\ 
  \hline 
  \\
\end{tabular}

\conceptsubsection{Expression semantics}

\T\begin{tabular}{p{2cm}p{3cm}lp{3cm}l}  \\ \hline
\W\begin{tabular}{lllll} \\  \hline 
  \bf  Name     &
  \bf  Expression &
  \bf  Precondition&
  \bf  Semantics &
  \bf  Postcondition
  \\ 
  \hline
  Anchor grid reference &
  {\tt grid\_type const\& g = e.TheGrid()}  &
  {\tt e} is valid &
  equivalent to  {\tt g = e.TheGrid()} &
  ~  \\ 
  \hline
  \\
\end{tabular}

\W\conceptsubsection{Complexity guarantees}

\conceptsubsection{Refinements}
\sectionlink{\concept{Vertex}}{GridVertex}
\\
\sectionlink{\concept{Edge}}{GridEdge}
\\
\sectionlink{\concept{Face}}{GridFace}
\\
\sectionlink{\concept{Facet}}{GridFacet}
\\
\sectionlink{\concept{Cell}}{GridCell}
\\

\W\conceptsubsection{Models}


\conceptsubsection{Notes}
\begin{enumerate}
\item 
  \notelabel{note-dim-codim}
  some of these types can coincide: for a concrete 2D-grid, 
  the types {\tt Edge}
  and {\tt Facet} can be the same. 
  But it is also possible to define  them as distinct types.

\item     
  \notelabel{note-face}
  There cannot be a  type {\tt Face} defined for 1D-grids.

\end{enumerate}

\conceptsubsection{See also}

\sectionlink{\concept{Grid}}{Grid} ~
\sectionlink{\concept{Grid  Entity}}{GridEntity} ~
\sectionlink{\concept{Grid Element Handle}}{GridElementHandle} 

    
  

