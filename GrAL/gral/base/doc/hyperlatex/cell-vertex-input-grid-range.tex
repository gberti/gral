    \xname{Cell-VertexInputGridRange}
\begin{Label}{Cell-VertexInputGridRange}
\conceptsection{Cell-Vertex Input Grid Range}
\end{Label}

\conceptsubsection{Description}
A {\em Cell-Vertex Input Grid Range\/} is a Grid Range which can be used to construct
another grid. More precisely, its representation is cell-based:
Supported is mostly iteration over cells and over vertices incident to cells.

\conceptsubsection{Refinement of}
\sectionlink{\concept{ Cell Grid Range }}{CellRange}
\\
\sectionlink{\concept{ Vertex Grid Range }}{VertexRange}

\conceptsubsection{Associated types}\noteref{note-gridtypes}

\begin{tabularx}{12cm}{llR} \\
  \hline
  \bf  Name  &\bf  Expression  &\bf  Description   \\
  \hline
  cell type &
  {\tt R::Cell} &
  model of \sectionlink{\concept{ Grid Cell }}{GridCell} 
  defining 
  \par {\tt R::Cell::VertexIterator}
  \par (a model of \sectionlink{\concept{VertexOnCellIterator}}{Vertex-On-CellIterator}) 
  \\ 
  \hline
  \\
\end{tabularx}

\begin{ifhtml}

\conceptsubsection{Valid Expressions}

None, besides those defined in 
\sectionlink{\concept{ Cell   Grid Range }}{CellGridRange}
and
\sectionlink{\concept{ Vertex Grid Range }}{VertexGridRange}

\conceptsubsection{Expression semantics}
None, besides those defined in 
\sectionlink{\concept{ Cell   Grid Range }}{CellGridRange}
and 
\sectionlink{\concept{ Vertex Grid Range }}{VertexGridRange}

\end{ifhtml}

\conceptsubsection{Models}
\sectionlink{{\tt enumerated\_grid\_range}}{EnumeratedGridRange}
\\
\sectionlink{{\tt Complex2D}}{Complex2D}
\\
\gralclasslink{IstreamComplex2DFmt}{base}
    
\W\conceptsubsection{Notes}

\conceptsubsection{See also}
\sectionlink{\concept{ Grid Range }}{GridRange} ~ 
\sectionlink{\concept{ Cell   Grid Range }}{CellGridRange} ~
\sectionlink{\concept{ Vertex Grid Range }}{VertexGridRange} ~
\sectionlink{\concept{ VertexOnCellIterator }}{Vertex-On-CellIterator}
  

