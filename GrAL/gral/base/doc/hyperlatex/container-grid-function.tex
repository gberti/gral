\xname{ContainerGridFunction}
\begin{Label}{ContainerGridFunction}
\conceptsection{Container Grid Function Concept}
\end{Label}

\conceptsubsection{Description}

The  {\em Container Grid Function\/} concept refines the 
\sectionlink{\concept{ Mutable Grid Function}}{MutableGridFunction} concept.
A Container Grid Function can be created and filled with values by a client,
much like a ordinary container. This of particular importance for 
algorithms needing temporary storage, such as boolean flags on grid elements.

\conceptsubsection{Refinement of}
\sectionlink{\concept{ Mutable Grid Function}}{MutableGridFunction}
\\
STL \xlink{Assignable}{http://www.sgi.com/Technology/STL/Assignable.html}

\conceptsubsection{Notation}
{\tt F} is a type which is a model of  Container Grid  Function 
\\
{\tt f} is an object of type  {\tt F}
\\
{\tt G} is shorthand for  {\tt F::grid\_type}
\\
{\tt g} is an object of type  {\tt G}.

\conceptsubsection{Associated types}
None, exept those defined in
\sectionlink{\concept{ Mutable Grid Function}}{MutableGridFunction}

\conceptsubsection{Valid Expressions}
\begin{tabular}{llll} 
  \hline
  \bf  Name  &\bf  Expression  &\bf  Type requirements  & \bf  return type  \\ \hline
  Default construction & 
  {\tt F f();} &
  ~ &
  ~ 
  \\ 
  Construction from grid & 
  {\tt F f(g);} &
  ~ &
  ~ 
  \\ 
  Construction and initialization & 
  {\tt F f(g,t);} &
  ~ &
  ~ 
  \\ 
  Binding to grid &
  {\tt f.set\_grid(g);} &
  ~ &
  ~ 
  \\ 
  \hline
\end{tabular}

\conceptsubsection{Expression semantics}
\begin{tabularx}{15cm}{RRRRR} 
  \hline    
  \bf  Name     &
  \bf  Expression &
  \bf  Precondition&
  \bf  Semantics &
  \bf  Postcondition
  \\ 
  \hline
  Default construction & 
  {\tt F f();} &
  ~ &
  default construct {\tt f} &
  {\tt f} is unbound
  \par write access is an error
  \par read  access is an error
  ~
  \\ 
  construction from grid & 
  {\tt F f(g);} &
  ~ &
  construct and bind {\tt f}  to {\tt g} &
  {\tt f} is bound to {\tt g} 
  \par write access is allowed 
  \par read access is undefined
  \\ 
  construction and initialization & 
  {\tt F f(g,t);} &
  ~ &
  construct and bind {\tt f}  to {\tt g}, 
  initialize all values to {\tt t} &
  {\tt f} is bound to {\tt g} 
  \par write access is allowed 
  \par {\tt f(e)} is equal to {\tt t} for all elements {\tt e}
  in the range of {\tt f}.
  \\ 
  Binding to grid &
  {\tt f.set\_grid(g);} &
  f is unbound &
  bind {\tt f}  to {\tt g} &
  {\tt f} is bound to {\tt g} 
  \par write access is allowed, 
  \par read access is undefined
  \\ 
  \hline
  \\
\end{tabularx}

\conceptsubsection{Complexity Guarantees}
Default construction takes constant time.
\\
Construction from grid and construction with initalization both
take time at most O({\tt f.size()}), that is, the number of 
elements of type {\tt F::element\_type} of {\tt g}.
 

\conceptsubsection{Refinements}
\sectionlink{\concept{ Total Grid Function}}{TotalGridFunction} ~
\W\\
\sectionlink{\concept{ Partial Grid Function}}{PartialGridFunction}

\W\conceptsubsection{Models}
    

\W\conceptsubsection{Notes}

 
\conceptsubsection{See also}
\sectionlink{\concept{ Grid Element Function }}{GridElementFunction} ~
\sectionlink{\concept{ Grid  Function }}{GridFunction} ~
\sectionlink{\concept{ Mutable Grid  Function }}{MutableGridFunction} ~

  

