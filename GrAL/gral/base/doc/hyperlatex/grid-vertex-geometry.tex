\xname{VertexGridGeometry}
\begin{Label}{VertexGridGeometry}
 \conceptsection{Vertex Grid Geometry Concept} 
\end{Label}

\conceptsubsection{Description}
The \concept{Vertex Grid Geometry} concept is the weakest
\sectionlinkshort{grid geometry}{GeometricalLayer} concept.
It simply provides a mapping from grid vertices to points in some space.

The \concept{Mutable Vertex Grid Geometry} concept allows in addition
the assignment of vertex coordinates.

\conceptsubsection{Refinement of}

\conceptsubsection{Notation}
\type{Geo} is a type which is a model of \concept{Vertex Grid Geometry} \\
{\tt g} is an object of type \type{Geo}\\
{\tt v, v1, v2} are objects of a type \type{V} which is a model of \conceptlink{Grid Vertex}{GridVertex}.\\
{\tt q} is an object of {\tt Geo::coord\_type}

\W\conceptsubsection{Definitions}

\conceptsubsection{Associated types}

\noindent
\begin{tabularx}{14cm}{llR} 
  \\ \hline
  \bf  Name  & \bf  Expression  &\bf  Description   \\
  \hline
  grid type &
  {\tt Geo::grid\_type} &
  underlying grid type, model of \conceptlink{Vertex Grid Range}{VertexGridRange}.
  \\
   point type &
   {\tt Geo::coord\_type}&
   the geometric point type, 
   representation of the elements of the topological 
   space where the geometry lives
   \par model
   of STL \Stllink{Assignable}.
   \\
  \hline
\end{tabularx}

\conceptsubsection{Valid Expressions}

\noindent
\begin{tabular}{llll}
  \\ \hline
  \bf  Name  &\bf  Expression  &\bf  Type requirements  & \bf  return type  \\
  \hline
   get vertex coordinates & q = g.coord(v)  &  & \type{coord\_type const\&} \\
   set vertex coordinates & g.coord(v) = q & \type{Geo} is mutable & \type{coord\_type \&} \\
  \hline
\end{tabular}

\conceptsubsection{Expression semantics}

\noindent 
 \begin{tabularx}{15cm}{RlRRR} 
   \\ 
   \hline 
   \bf  Name    &
   \bf  Expression &
   \bf  Precondition&
   \bf  Semantics &
   \bf  Postcondition
   \\ 
   \hline
    coordinate read access & 
    {\tt q = g.coord(v)} &
    {\tt v} is \footlink{valid}{valid} &
    if {\tt v1 == v2} then {\tt g.coord(v1)} is the same as\noteref{note-gvg-comparison}
    {\tt g.coord(v2)} &
    \\
    coordinate write access &
    {\tt g.coord(v) = q} &
    {\tt v} is \link{valid}{valid} &
    set the coordinate of {\tt v} to {\tt q} &
    {\tt g.coord(v)} is equal to {\tt q}
    \\
   \hline
\end{tabularx}

\W\conceptsubsection{Invariants}

\conceptsubsection{Refinements}
\conceptlink{Volume Grid Geometry}{VolumeGridGeometry}

\conceptsubsection{Models}
\sectionlink{\type{IstreamComplex2DFmt}}{istream-complex2d-fmt}

\conceptsubsection{Notes}
\begin{enumerate}
\item \notelabel{note-gvg-comparison} 
In general, the type \type{Geo::coord\_type} will not be a model
of STL \stllink{Equality Comparable}{EqualityComparable},
because this is normally not useful for floating point values.
\end{enumerate}

\conceptsubsection{See also}


