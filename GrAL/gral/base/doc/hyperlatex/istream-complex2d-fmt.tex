\xname{istream-complex2d-fmt}
\begin{Label}{istream-complex2d-fmt}
\datasection{IstreamComplex2DFmt}
\end{Label}

\datasubsection{Declaration}
class \gralclasslink{IstreamComplex2DFmt}{base};

\datasubsection{Description}
The class \type{IstreamComplex2DFmt} is an adapter to the grid interface
for a grid given by a sequential stream, for example a file,
in the complex2d format.
By using this class, it is possible to read an arbitrary grid from
this format, without knowing its details.

\datasubsection{Model of}
\conceptlink{Cell-Vertex Input Grid Range}{Cell-VertexInputGridRange}
\\
\conceptlink{Vertex Grid Geometry}{VertexGridGeometry}

\datasubsection{Definition}
Defined in \gralfilelink{complex2d-format-input}{h}{base}
\datasubsection{Public base classes}
\datasubsection{Members}
\datasubsection{New members}
\begin{tabularx}{14cm}{lR}
  \hline
  \bf Member & \bf Description \\
  \hline
  \pcode{IstreamComplex2DFmt(istream\&, int offset = 0)} &
     Constructor, offset gives the lowest vertex number used
    (usually 0 or 1).
    \\
    \pcode{coord\_type const\& coord(Vertex const\& v)}
    & access to vertex coordinates.  
    \\
    \hline
\end{tabularx}

\datasubsection{Example}
\begin{example}
a_grid_type Agrid;
a_geom_type Ageom; // model of \conceptlinkfoot{Mutable Grid Vertex Geometry}{VertexGridGeometry}
ifstream in("grid.complex2d");
IstreamComplex2DFmt SrcG(in); // Grid & geom combined

// (Agrid, Ageom) = (SrcG, SrcG)
\conceptlinkfoot{Construct}{ConstructGrid}(Agrid, Ageom, SrcG, SrcG);
\end{example}
\datasubsection{Known uses}
\datasubsection{Notes}
\datasubsection{See also}

