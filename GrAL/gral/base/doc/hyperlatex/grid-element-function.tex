\xname{GridElementFunction}
\begin{Label}{GridElementFunction}
  \conceptsection{Grid Element Function Concept}
\end{Label}

\conceptsubsection{Description}
The {\em Grid Element Function\/} concept 
models the mathematical concept of a mapping from
grid elements of some fixed type (vertex, edge, cell)
which is a model of \sectionlink{\concept{Grid Element}}{GridElement},
to values of  some type {\tt T}.

\conceptsubsection{Refinement of}
STL \stllink{Adaptable Unary Function}{AdaptableUnaryFunction}

\conceptsubsection{Notation}
{\tt F} is a type which is a model of  Grid Element Function 
\\
{\tt f} is an object of type  {\tt F}
\\
{\tt e, e1, e2} are objects of  {\tt F::element\_type}
\\
{\tt t} is an object of  {\tt F::value\_type}

\conceptsubsection{Associated types}

\noindent
\begin{tabularx}{14cm}{llR}  
  \hline
  \bf  Name  &\bf  Expression  &\bf  Description   \\ 
  \hline
  element type  & 
  {\tt F::element\_type} &
  type of the underlying element, 
  \par model of \sectionlink{\concept{ Grid Element}}{GridElement} 
  \par synonym for {\tt F::argument\_type}
  (from \stllink{Adaptable Unary Function}{AdaptableUnaryFunction})
  \\
  value type  &
  {\tt F::value\_type} &
  synomym for {\tt F::result\_type}   
  \par (from \stllink{Adaptable Unary Function}{AdaptableUnaryFunction})
  \\ 
  \hline
  \\
\end{tabularx}
    
\conceptsubsection{Valid Expressions}
No new ones, besides those from 
\stllink{Adaptable Unary Function}{AdaptableUnaryFunction}

\noindent
\begin{tabular}{llll} \\
  \hline
  \bf  Name  &\bf  Expression  &\bf  Type requirements  & \bf  return type  \\ 
  \hline
  function evaluation  &
  {\tt t = f(e);} &
  ~      &
  {\tt value\_type} 
  \\ 
  \hline
  \\
\end{tabular}

\conceptsubsection{Expression semantics}
The semantics of function evaluation are more restrictive than those for
\stllink{Adaptable Unary Function}{AdaptableUnaryFunction}
see the notes below\noteref{note-evaluation}.

\noindent
\begin{tabularx}{14cm}{llRRR} \\
  \hline
  \bf  Name       &
  \bf  Expression &
  \bf  Precondition&
  \bf  Semantics &
  \bf  Postcondition
  \\ 
  \hline
  evaluation  &
  {\tt t = f(e)} &
  {\tt e} is in the domain of {\tt f} &
  evaluate {\tt f} at the argument {\tt e} & 
  {\tt t} is equal to {\tt f(e)}\noteref{note-evaluation}
  \\ 
  \hline
  \\
\end{tabularx}

\conceptsubsection{Invariants}
\begin{tabular}{ll} 
  Argument identity &
  if {\tt e1 == e2} then {\tt f(e1)} is equal to {\tt f(e2)} \noteref{note-equal-comp}
  \\ 
\end{tabular}

\conceptsubsection{Refinements}
\sectionlink{\concept{ Grid Function }}{GridFunction}

\conceptsubsection{Models}
\sectionlink{{\tt cell\_nb\_degree<GRID>}}{cell_nb_degree}
defined in
\gralfilelink{grid-functors}{h}{base}.

\conceptsubsection{Notes}
\begin{enumerate}
\item \notelabel{note-evaluation}
  The important difference to STL function objects
  is that the latter are {\em not\/} guaranteed to deliver the same result 
  for subsequent evaluations on the same argument.
\item \notelabel{note-equal-comp}    
  The type {\tt F::value\_type} is not required to be 
  STL \stllink{Equality Comparable}{EqualityComparable}
  If it is, then {\tt e1 == e2} implies {\tt f(e1) == f(e2)}.
\end{enumerate}
    
\conceptsubsection{See also}
\sectionlink{\concept{ Grid Function }}{GridFunction} ~

  

