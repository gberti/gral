\xname{Grid}
\begin{Label}{Grid}
\conceptsection{Grid Concept}
\end{Label}

\conceptsubsection{Description }
The  mathematical concept underlying {\em  Grid }    
is that of (a subset of) a (finite)     
\xlink{CW-complex}[\cite{weingram}]{http://www.treasure-troves.com/math/CW-Complex.html}. 
Some well-known specialization of this concept are    
triangulations, boundary complexes of convex  
\xlink{polytopes}{http://www.treasure-troves.com/math/Polytope.html} 
and Cartesian grids.     

\conceptsubsection{Refinement of }
\sectionlink{\concept{Grid Range}}{GridRange}  

The main difference to grid ranges is that grids stand for their own ---
there is no underlying base grid. 
This means that all grid entities produced by calls to member functions of a grid      
{\tt  g} refer to {\tt  g} with their grid anchor references.    

Virtually all algorithms can do with grid ranges, they do not require grids.    

\conceptsubsection{Notation }
{\tt  G} is a type which is a model of grid      

\conceptsubsection{Associated types}
    
\begin{tabular}{ccc} \\ 
  \hline
  {\bf  Name  } & {\bf  Expression  } & {\bf  Description  } \\ 
  \hline 
  base grid  & {\tt  G::grid\_type}  & identical to {\tt  G }  \\ 
  \hline
  \\
\end{tabular}


\conceptsubsection{Refinements}
\conceptlink{Grid-With-Boundary}{Grid-With-Boundary}
\conceptsubsection{Models }
\sectionlinkUNDEF{\type{Triang2D}}{Triang2D}
\\
\sectionlink{\type{Complex2D}}{Complex2D.html}   

\conceptsubsection{Notes}
 \begin{enumerate}
   \item Technically, these types are bundled in a struct 
     {\tt  grid\_types<G>}      
     which is used by the algorithms to access these types. 
     This opens up the possibility  to parameterize algorithms by such a {\em  traits class } 
     like {\tt  grid\_types<G> },      
     thereby introducing different iterator and element types, 
     for example counting iterators or debug iterators producing graphical output.    
\end{enumerate}

\conceptsubsection{See also }
\sectionlink{\concept{Grid  Range}}{GridRange} ~
\sectionlink{\concept{Grid Element}}{GridElement} ~ 
\sectionlink{\concept{Grid Sequence Iterator}}{GridSequenceIterator} 
