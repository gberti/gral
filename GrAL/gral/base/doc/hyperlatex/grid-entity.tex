\xname{GridEntity}
\begin{Label}{GridEntity}
\conceptsection{Grid Entity Concept}
\end{Label}

\conceptsubsection{Description}
The {\em Grid Entity\/} concept is rather abstract in that it gives rise to rather
separate sub-concepts, notably \sectionlink{\concept{ Grid Element }}{GridElement},
\sectionlink{\concept{Grid Sequence Iterator}}{GridSequenceIterator},
and \conceptlink{Grid Incidence Iterator}{GridIncidenceIterator}.
Its purpose is to bundle some fundamental properties, much like the STL concepts
\Stllink{Assignable} and
\stllink{Equality Comparable}{EqualityComparable}
which it refines.

The fundamental property captured by {\em Grid Entity\/} is binding to a 
\sectionlink{\concept{Grid}}{Grid}
(the {\em anchor grid\/}), 
and global identity, 
in contrast to \sectionlink{\concept{Element Handles}}{GridElementHandle},
which are unique only within a single grid.

Conceptually, a grid is the owner of its entities; 
an entities lifetime cannot exceed that of
its anchor grid. 
Also, a grid entity needs  not necessarily to be stored permanently,
it may be constructed only temporarily, 
for example a
\concept{Grid Iterator}
or a  
\sectionlink{\concept{Grid Element}}{GridElement}
obtained by dereferencing a  Grid Iterator.

\conceptsubsection{Refinement of}
STL \Stllink{Assignable}
\\
STL \stllink{Equality Comparable}{EqualityComparable}


\conceptsubsection{Notation}
{\tt E} is a type which is a model of {\em Grid Entity\/}
\\
{\tt e, e1, e2} are objects of type {\tt E}

\conceptsubsection{Definitions}

      \label{anchor}
      An {\em anchor\/} of {\tt e} is a logically superior entity.
      It is either an object of a type which is a model of 
      \sectionlink{\concept{Grid}}{Grid}
      or an object of a type which is a model of 
      \sectionlink{\concept{Grid Element}}{GridElement},
      to which {\tt e} is incident. 
      This is defined more precisely depending on the
      subconcept involved. For example, for
      \sectionlink{\concept{ Grid Element}}{GridElement}
      the anchor is the underlying grid;
      for a \sectionlink{\concept{Grid Sequence Iterator}}{GridSequenceIterator}
      on a 
      \sectionlink{\concept{Grid Range}}{GridRange},
      it is the grid range, 
      and for a
      \sectionlink{\concept{Grid Incidence Iterator}}{GridIncidenceIterator},
      it is a grid element.
      So, the anchor of a 
      \sectionlink{\concept{Vertex}-On-Cell Iterator}{Vertex-On-CellIterator},
      is the cell over whose incident vertices the iteration is performed.
 
    \label{valid}
    
     A grid entity {\tt e} is {\em valid\/}, 
      if its anchor is valid (either refers to an existing grid, or is itself a valid Grid Entity),
      and its handle is a valid handle of that grid.


\conceptsubsection{Associated types}

\begin{tabular}{lll} 
  \T \\ \hline
  \bf  Name  & \bf  Expression  &\bf  Description   \\ \hline
  Grid type &  {\tt E::grid\_type} &
  type of the corresponding \sectionlinkshort{anchor grid}{Grid}   \\ 
  Anchor type &  {\tt E::anchor\_type} &
  type of the corresponding \footlink{anchor}{anchor}   \\ 
  handle type &  {\tt E::handle\_type} &
  type of the corresponding \sectionlinkshort{element handle}{GridElementHandle}  
  \T \\ \hline \\
\end{tabular}



\conceptsubsection{Valid Expressions}
\begin{tabular}{llll}
  \T \\ \hline
  \bf  Name  &\bf  Expression  &\bf  Type requirements  & \bf  return type  \\ \hline
  Anchor grid &  {\tt e.TheGrid()} & 
   ~ & 
  {\tt grid\_type const\&} 
  \\ 
  Anchor entity &  {\tt e.TheAnchor()} & 
  ~ & 
   {\tt anchor\_type const\&} 
   \\
   Handle &  {\tt e.handle()} & 
   ~ & 
   {\tt handle\_type}
   \T \\ \hline   \\
 \end{tabular}
 
 \conceptsubsection{Expression semantics}
 
 %\begin{tabular}{|p{2cm}|l|p{4cm}|p{4cm}|p{4cm}} \hline
 \begin{tabularx}{15cm}{XlXXX} 
   \T \\ \hline 
   \bf  Name    &
   \bf  Expression &
   \bf  Precondition&
   \bf  Semantics &
   \bf  Postcondition
   \\ \hline
   Copy constructor &
   {\tt E e1(e2)} &
   ~    &
   ~    &
   {\tt e1 == e2} 
   \\ 
   Assignment &
   {\tt e1 = e2} &
   ~    &
   {\tt e1} is a copy of {\tt e2}       &
   {\tt e1 == e2}
   \\ 
   Anchor grid  &
   {\tt e.TheGrid()} &
   {\tt e} is \link{valid}[~(section \Ref)~]{valid} &
   get the grid to which {\tt e} is bound   & 
   ~  
   \\ 
   Equality comparison &
   {\tt e1 == e2} &
   {\tt e1.TheAnchor() == e2.TheAnchor()} \noteref{note-anchor} &
     true if {\tt e1} and {\tt e2}
     refer to the same element of  {\tt e1.TheGrid()} 
     {\tt e1 == e2} is equivalent to
     {\tt e1.handle() == e2.handle()} 
     and 
     {\tt e1.TheAnchor() == e2.TheAnchor()}\noteref{note-grid-comparison}
  & ~
   \T \\ \hline  \\ 
 \end{tabularx}

 \conceptsubsection{Refinements}
 \sectionlink{\concept{Grid Element}}{GridElement}
 \\ 
 \sectionlink{\concept{Grid Sequence Iterator}}{GridSequenceIterator}
 \\ 
 \sectionlink{\concept{Grid Incidence Iterator}}{GridIncidenceIterator}
 \\ 

 \conceptsubsection{Notes}

 \begin{enumerate}
 \item 
   \notelabel{note-grid-comparison}
   This does not imply that for grids, there is a real equality comparison required.
   Rather, equality of grids generally means  {\em physical\/} identity of the corresponding
   objects, that is, their addresses.
   In contrast, equality for Grid Entities generally means {\em logical\/} rather than physical
   identity, because they might be created only temporarily.
   
   Put another way, for grid entities, {\tt e1 = e2} entails {\tt e1 == e2},
   which is not the case for grids.

 \item
   \notelabel{note-anchor}
   It can be discussed if this precondition is necessary. 
   Formally, two entities
   compare as different if not {\tt e1.TheAnchor() == e2.TheAnchor()}.
   However, requiring this equality allows to compare only the handles, 
   which is  faster.
 \end{enumerate}

 \conceptsubsection{See also}
 
 \sectionlink{\concept{Grid}}{Grid} ~
 \sectionlink{\concept{Element Handle}}{GridElementHandle}

      
  

